\vspace{-.6cm}
{\large {
    Este documento presenta la solución final al proyecto de la materia de Computación Distribuida. El objetivo de este trabajo es desarrollar una blockchain para un sistema de criptomonedas, implementado en el lenguaje de programación \textit{Elixir}. \vspace{.6cm}

    Durante la ejecución del proyecto, se simulará el sistema de criptomonedas gestionando los distintos procesos distribuidos que representan a los usuarios. Estos usuarios realizarán transacciones entre ellos, enviando mensajes para alcanzar un consenso y detectando y eliminando intentos de alteración de la cadena de bloques. La blockchain, aunque simula un entorno, debe operar de forma descentralizada y garantizar la integridad de la información, asegurando que los nodos honestos mantengan una copia consistente de las transacciones para que estas se validen e inserten correctamente. \vspace{.6cm}

    Para modelar las conexiones de la red, se utiliza una implementación basada en el modelo \textit{Watts y Strogatz}, que garantiza un coeficiente de agrupamiento alto y una estructura que favorece la propagación eficiente de la información. Se asegura que este coeficiente sea mayor a 0.4 para comenzar el proceso de consenso y transmisión de transacciones. \vspace{.6cm}

    Una de las características importantes del proyecto es la introducción de \textbf{procesos} \textbf{bizantinos}, los cuales representan nodos maliciosos en la red. Estos procesos se encargan de generar bloques basura o incorrectos con la intención de alterar la blockchain y comprometer su integridad. Sin embargo, el sistema debe ser capaz de detectar y evitar que estos bloques maliciosos sean insertados en la cadena. Para garantizar la seguridad y la correcta operación de la red, se asume que, si hay $f$ nodos bizantinos y $n$ nodos en total, se cumple la condición $ n > 3f $, lo que asegura que la mayoría de los nodos honestos puedan mantener la coherencia de la blockchain. \vspace{.6cm}

    \textbf{Consideraciones extra}  \vspace{.6cm}

    Se pueden utilizar cuantas funciones auxiliares sean necesarias, siempre que se logre el objetivo deseado. No es necesario implementar la persistencia del estado o la validación de la cartera; una vez que el programa termina, la blockchain deja de existir, y no se almacenará nada de forma permanente. No se realizará ninguna verificación sobre la validez de las transacciones en cuanto a saldo disponible. Se proporcionará un módulo inicial llamado \textit{Crypto} encargado de realizar los hasheos de los bloques. Finalmente, la red debe ser capaz de evitar la inserción de bloques generados por procesos bizantinos, manteniendo así la integridad del sistema.
}}